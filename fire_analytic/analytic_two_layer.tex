\documentclass[12pt]{article}
\usepackage[parfill]{parskip}

\usepackage[utf8]{inputenc}

\title{NSF proposal}
\author{Octavia Crompton}
\date{July 2019}

% \usepackage[,inline]{trackchanges}
\usepackage{scrextend}

% \usepackage[numbers]{natbib}
\usepackage[sort&compress]{natbib}
\usepackage[margin=1in]{geometry}
\usepackage{caption}
% \captionsetup[figure]{font=small,labelfont=bf,textfont=bf}
% \captionsetup[figure]{font=small,labelfont=bf}
\captionsetup[figure]{font=small,labelfont=bf}
\usepackage{graphicx}
\usepackage{wrapfig}
\usepackage{sidecap}
% \usepackage{apacite}

% \usepackage[colorlinks]{hyperref}
\usepackage[dvipsnames]{xcolor}

\usepackage{amssymb,amsmath}
\usepackage{kpfonts,baskervald}
\usepackage{scalerel}
\usepackage{array}
\newcolumntype{L}[1]{>{\raggedright\let\newline\\\arraybackslash\hspace{0pt}}m{#1}}
\newcolumntype{C}[1]{>{\centering\let\newline\\\arraybackslash\hspace{0pt}}m{#1}}
\newcolumntype{R}[1]{>{\raggedleft\let\newline\\\arraybackslash\hspace{0pt}}m{#1}}


\usepackage[compact]{titlesec}
% \titlespacing{\section}{0pt}{*1.}{*1.5}
% \titlespacing{\subsection}{0pt}{*1.2}{*1.2}
% \titlespacing{\subsection}{10pt}{*0}{*0}


\usepackage{enumitem}
\definecolor{bluePigment}{rgb}{0.2, 0.2, 0.6}
\renewcommand{\figurename}{\color{bluePigment}{Figure}}

\definecolor{OliveGreen}{cmyk}{0.64,0,0.95,0.40}
\usepackage{float}
\renewcommand{\rmdefault}{phv} % Arial
\renewcommand{\sfdefault}{phv} % Arial

% \renewcommand{\doiprefix}{}
% \newcommand{\doi}[1]{https://doi.org/#1}

% \pagenumbering{gobble}

\begin{document}
% \maketitle




\subsection*{ Equations and assumptions}

The equations for the upper  and lower  canopies are:

\begin{equation}
    \frac{dG_u}{dt} =  r_{u} S^\beta G_u \big(1-\frac{G_u}{k_u}\big)
    \label{G_u}
\end{equation}

 \begin{equation}
    \frac{d G_l}{dt} = r_l S^\beta G_l \big(1-\frac{G_l}{k_l}\big) - \alpha G_u G_l,
        \label{G_l}
\end{equation}


where subscripts indicate the canopy level, $G$ is the biomass density, $r$ is the growth rate, $k$ is the carrying capacity, and $\alpha$ is a competition parameter, which effects the lower canopy only. 

A vegetation growth limiting factor, $S^\beta$, slows the biomass growth rates as soil moisture decreases; this effect increases with increasing $\beta$.


Assume that fires occur with fixed return interval $\xi$  and severity $\phi_S$. 

The biomass immediately before each fire  ($G_{u,max}$) and immediately after ($G_{uo}$) are related as:

\begin{equation}
    \frac{G_{u,max}}{G_{uo}} = \frac{1}{\phi_R}
\end{equation}

where $\phi_R$ is the proportion of biomass that remains after each fire.  Defining fire severity as $\phi_S = 1-\phi_R$:

\begin{equation}
    \frac{G_{u,max}}{G_{uo}} = 1 - \frac{1}{\phi_S}
\end{equation}


\subsection*{  Analytic solution for the upper canopy}


Equation \ref{G_u}  is the logistic equation, and  has an analytic solution:
\begin{equation}
    G_u = \frac{k_u G_{uo}}{G_{uo} +(k_u-G_{uo}) e^{-r'_u t}}
    \label{logistic_solution}
\end{equation}

where  $G_{uo}$  is the initial biomass, and $r'_u =  r_u S^\beta  $  combines the growth rate and growth-limiting factor into an `effective` growth rate.  

If time is measured as the time since the previous fire, then $G_u(t = 0) = G_{uo}$ and  $G_u(t = \xi) = G_{u,max}$  (once the system is in a dynamic steady state):

\begin{equation}
    \frac{G_{u,max}}{G_{uo}} = \frac{1}{\phi_R} =  \frac{k_u }{G_{uo} +(k_u-G_{uo}) e^{-r'_u \xi}}
\end{equation}

 Solving for $G_{uo}$:
 
\begin{equation}
  G_{uo} =  k_u \  \frac{\phi_R - e^{-r'_u \xi} }{1 - e^{-r'_u \xi}}
\end{equation}

Or, in terms of severity:

\begin{equation}
  G_{uo} =  k_u \   \frac{1- \phi_S - e^{-r'_u \xi} }{1 - e^{-r'_u \xi}}
  \label{G_uo}
\end{equation}

\subsection*{Mean upper canopy biomass}

Integrate Equation \ref{logistic_solution} to get average $G_u$:

\begin{equation*}
 \int_{G_{uo}}^{G_{u,max}} G_u dt = 
 \int_{0}^\xi \frac{k_u G_{uo}}{G_{uo} +(k_u-G_{uo}) e^{- r'_u t}}dt
\end{equation*}

\begin{equation*}
 \quad  =\frac{k_u}{r'_u}\log \big(| G_{uo} - G_{uo} e^{r'_u\xi}  - k_u | \big)
  - \frac{k_u}{r'_u}\log \big(|  - k_u | \big),
\end{equation*}


which simplifies to:

\begin{equation*}
 \int_{G_{uo}}^{G_{u,max}} G_u dt =
 		 \frac{k_u}{r'_u}\log \big(1 + \frac{G_{uo}}{k_u}( e^{r'_u \xi}-1)\big)
\end{equation*}

Once the system is in dynamic equilibrium, the mean upper canopy biomass $\hat G_u$ is:

\begin{equation}
\hat{G}_u =
 		 \frac{k_u}{r'_u \xi}\log \bigg(1 + \frac{G_{uo}}{k_u}( e^{r'_u \xi}-1)\bigg)
\end{equation}

Substitute Equation  \ref{G_uo} and simplify:

\begin{equation}
\hat{G}_u =
 \frac{k_u}{r'_u \xi}\log \big(1 +   \frac{1- \phi_S - e^{-r'_u \xi} }{1 - e^{-r'_u \xi}} ( 1 - e^{-r'_u \xi})\big)
\end{equation}

    
\begin{equation}
\hat{G}_u =
  k_u \big( 1 + \frac{1}{r'_u \xi} \log(1-\phi_S) \big)
		\label{G_u_mean}
\end{equation}


\subsection*{Stability: conditions to sustain upper canopy biomass}

Requiring that $\hat{G}_u >0$ in Equation \ref{G_u_mean} yields:

\begin{equation}
\phi_S < 1- e^{-r'_u \xi}
\end{equation}

In terms of return time:

\begin{equation}
\xi > - \frac{1}{r'_u}\log (1 - \phi_S) 
\end{equation}

For lower growth rates, a longer return interval is needed to sustain biomass.
If $\beta>0$, a longer return time is needed for lower soil moisture conditions:

\begin{equation}
\xi > - \frac{1}{r_u S^\beta}\log (1 - \phi_S)
\end{equation}

Rearranging to find the minimum soil moisture to sustain the upper canopy:
 
\begin{equation}
S > \bigg( - \frac{1}{r_u \xi }\log (1 - \phi_S)\bigg)^{1/\beta}
\end{equation}

\begin{equation}
\hat{G}_u  =
    \begin{cases}
       k_u \big( 1 + \frac{1}{r'_u \xi} \log(1-\phi_S) \big)
	  & \text{if  } S > \big( - \frac{1}{r_u \xi }\log (1 - \phi_S)\big)^{1/\beta}
		\\[10pt]
      0 & \text{otherwise}
    \end{cases}       
\end{equation}




\subsection*{Return interval and severity are related}

Suppose $\hat G_u = \gamma k_u$, where $\gamma<1$. From  Equation \ref{G_u_mean}:

\begin{equation}
\gamma =  1 + \frac{1}{r'_u \xi} \log(1-\phi_S) 
\end{equation}

\begin{equation}
\xi = -\frac{1}{r'_u (1-\gamma) }\log(1-\phi_S)
\end{equation}

Rearranging for $\phi_S$: 

\begin{equation}
\phi_S = 1- e^{-\xi r'_u (1-\gamma)}
\end{equation}

\textbf{Take-away}
 The relationship between $\hat G_u$ and $k_u$ constrains that between $RI$ and severity, and vice versa.


\subsection*{A modified logistic equation for the lower canopy}

The equation for the lower canopy  is:

 \begin{equation}
    \frac{d G_l}{dt} = r_l S^\beta G_l \bigg(1-\frac{G_l}{k_l}\bigg) - \alpha G_u G_l
\end{equation}

Since we have an analytic solution for $\hat G_u$,  we can modify the logistic equation to solve for $\hat G_l$.
Rewrite $G_l$  in logistic equation form:

  \begin{equation}
    \frac{d G_l}{dt} = r'_l \bigg(1-\frac{G_l}{k'_l}\bigg)
\end{equation}

where $r'_l = r_l S^\beta - \alpha \hat G_u$ and $k'_l  = k_l r'_l / r_l S^\beta$.

The analytic solution for $G_{uo}$ (Equation \ref{G_uo}) can then be used to estimate $G_l$ after the fire:

\begin{eqnarray}
  G_{lo} =  k'_l \   \frac{1- \phi_S - e^{-r'_l \xi} }{1 - e^{-r'_l \xi}}
\end{eqnarray}

 Similarly, the analytic solution for the mean  biomass (Equation \ref{G_u_mean}) can be used to estimate  $\hat G_l$:

\begin{equation}
\hat{G}_l =
 		k'_l \bigg( 1 + \frac{1}{r'_l \xi} \log(1-\phi_S) \bigg)
\label{hatG_l}
\end{equation}
    
Substitute $\hat G_u$ (Equation \ref{G_u_mean}) into $r'_l = r_l S^\beta - \alpha \hat G_u$ to get the modified growth rate:
    
\begin{equation}
r'_l= r_l S^\beta   - \alpha k_u \big( 1 + \frac{1}{r'_u \xi} \log(1-\phi_S) \big)
\end{equation}
   
And similarly for  $k'_l  = k_l r'_l / r_l S^\beta$:
    
\begin{equation}
k'_l= \frac{k_l}{r_l S^\beta } \bigg[  r_l S^\beta   - \alpha k_u \big( 1 + \frac{1}{r'_u \xi} \log(1-\phi_S) \big) \bigg]
\end{equation}

Which simplifies to:

\begin{equation}
k'_l=  k_l  \bigg[ 1 -  \frac{  \alpha k_u} {r_l S^\beta }\big( 1 + \frac{1}{r'_u \xi} \log(1-\phi_S) \big) \bigg]
\end{equation}   

\subsection*{How does biomass change with decreasing soil moisture?}

The derivative of $\hat G_u$ with respect to soil moisture is:

\begin{equation}
\frac{ d\hat{G}_u }{dS} =
    \begin{cases}
      \frac{-\beta k_u }{r_u \xi} \log(1-\phi_S) S^{-(1+\beta)}
		& \text{if  } S > \big( - \frac{1}{r_u \xi }\log (1 - \phi_S)\big)^{1/\beta}
		\\[10pt]
      0 & \text{otherwise}
    \end{cases}       
\end{equation}


For the lower canopy, the derivative is:

\begin{equation}
\frac{ d\hat{G}_l }{dS} =
    \begin{cases}
	  \frac{\beta  {k_l} }{{r_l} {r_u}} S^{-2 \beta -1}
\big( {r_u} S^{\beta }(\alpha  {k_u}  - Z) +2 Z \alpha  {k_u} 
   \big)
	  & \text{if  } S > \big( - \frac{1}{r_l \xi }\log (1 - \phi_S)\big)^{1/\beta}
		\\[10pt]
        \frac{-\beta k_l }{r_l \xi} \log(1-\phi_S) S^{-1-\beta} & \text{otherwise}
    \end{cases}       
\end{equation}


where

\begin{equation}
Z = \frac{\log({1-\phi_S})}{\xi}
\end{equation}

 $\frac{d \hat{G}_l}{dS}$  has a minimum value where 
  
  \begin{equation}
  \big( {r_u} S^{\beta }(\alpha  {k_u}  - Z) +2 Z \alpha  {k_u} 
   \big) = 0
\end{equation}

Rearranging:

  \begin{equation}
 S^{\beta } = \frac{-2 Z \alpha  {k_u} }{{r_u}(\alpha  {k_u}  - Z)  }
\end{equation}

Using the definition of $Z$:

  \begin{equation}
 S^{\beta } = \frac{-2 {\log({1-\phi_S})} \alpha  {k_u} }{{r_u}(\alpha  {k_u} \xi - {\log({1-\phi_S})})  }
\end{equation}

The numerator and denominator are both always positive (because $- \log(1-\phi_S) > 0$), so we expect a minima in $\hat G_l$ for $\alpha > 0$.

 $\frac{d \hat{G}_l}{dS}$ has a discontinuity at  $ S = \big( - \frac{1}{r_l \xi }\log (1 - \phi_S)\big)^{1/\beta}$, where $G_l$ has a maximum value.   


\subsection*{Maximum return interval for a lower canopy}

Equation \ref{hatG_l} can be used to estimate the conditions for which $\hat G_l >0$:

\begin{equation}
\hat{G}_l =
 		k'_l \big( 1 + \frac{1}{r'_l \xi} \log(1-\phi_S) \big) > 0
\end{equation}


\begin{equation}
 		\frac{1}{r'_l \xi} \log(1-\phi_S) > -1
\end{equation}

With some math:

\begin{equation}
 	\xi > - \frac{\log(1-\phi_S)}{r_u S^\beta} \frac{\alpha k_u - r_u S^\beta}{\alpha k_u - r_l S^\beta}
\end{equation} 


  \subsection*{$\hat G_l$ along a `stability' curve?}
  
What happens if we move  $G_l$ along the `stability' curve, $\phi_S = 1- e^{-\xi r'_u (1-\gamma)}$?

This expression can be re-witten as: 

 $\frac{\log(1 - \phi_S)}{  r'_u \xi  }= -  (1 - \gamma )$
 
 
 Substituting into the equations for $r'_l$ and $k'_l$:

\begin{equation}
r'_l= r_l S^\beta   - \alpha k_u \gamma
\end{equation}


\begin{equation}
 k'_l  = k_l  \bigg(1  - \frac{\alpha k_u \gamma} { r_l S^\beta}\bigg)
\end{equation}   


Substituting into the Equation for $\hat G_l$:

\begin{equation}
\hat{G}_l =
 		k'_l \big( 1 - \frac{(1 - \gamma) r'_u }{r'_l }  ) \big)
\end{equation}



\begin{equation}
\hat{G}_l =
 		  k_l  \bigg(1  - \frac{\alpha k_u \gamma} { r_l S^\beta}\bigg) 
		  \bigg( 1 - \frac{(1 - \gamma) r_u  S^\beta }{(r_l S^\beta   - \alpha k_u \gamma)}   \bigg)
\end{equation}


\begin{equation}
\hat{G}_l =
 		  {k_l}
		  \bigg( 1 - \frac{\alpha k_u \gamma - (1 - \gamma) r_u  S^\beta}{r_l S^\beta }  \bigg)
\end{equation}


\textbf{Take-away}
 The relationship between $\hat G_u$ and $k_u$ also determines how $\hat G_l$ depends on RI and severity.
 
Above, with $G_l= \gamma k_u$,  $\hat{G}_l$ does not depend on return interval or severity.

$$
\frac{\alpha  {k_l} {k_u} ({k_l} ({r_u}+z)-\chi 
   {G_l} {r_u})-{r_u} (\chi  {G_l} {r_l}
   ({G_l}-{k_l})+{k_l} ({k_l}
   ({r_l}+z)-{G_l} {r_l}))}{{r_l} {r_u} (\chi
    {G_l}-{k_l})} = 0
   $$
   
   
   
$$
{\alpha  {k_l} {k_u} ({k_l} ({r_u}+z)-\chi 
   {G_l} {r_u})-{r_u} (\chi  {G_l} {r_l}
   ({G_l}-{k_l})+{k_l} ({k_l}
   ({r_l}+z)-{G_l} {r_l}))}
 = 0
   $$

\subsection*{Modify ignition probability to decrease with increasing $S$}

Suppose the probability of ignition is: $p = \frac{1-S}{\xi}$.

The effective return time will be: $\xi' = \frac{\xi} { 1- S}$, and the equation for $\hat{G}_u $ becomes:

\begin{equation}
\hat{G}_u =
  k_u \big( 1 + \frac{(1-S)}{r_u S^\beta \xi} \log(1-\phi_S) \big)
\end{equation}

where $\xi$ is now interpretable as the return interval where $S=1$.

 \subsection*{ $G_u$ approaches $k_u$  leading up to each fire. How close does it get?}

The  pre-fire biomass is:

\begin{equation}
  G_{u,max} =  \frac{k_u }{1-\phi_S}   \frac{1- \phi_S - e^{-r'_u \xi} }{1 - e^{-r'_u \xi}}
  \label{G_u_max}
\end{equation}

Expanding, we have:

\begin{equation}
  G_{u,max} =  k_u  \frac{1- \phi_S - e^{-r'_u \xi} }{1  - \phi_S - e^{-r'_u \xi} + \phi_S  e^{-r'_u \xi}}
\end{equation}


Define $x$ as:

\begin{equation}
x = \frac{ \phi_S  e^{-r'_u \xi}}{1- \phi_S - e^{-r'_u \xi}}
\end{equation}

so $ G_{u,max} = k_u /(1+x)$:

\begin{equation}
G_{u,max}  = k_u \frac{1}{1 + \frac{ \phi_S  e^{-r'_u \xi}}{1- \phi_S - e^{-r'_u \xi}}}
\end{equation}


$G_{u,max} \sim k_u$ when $x$ is very small. 
Suppose we want $G_{u,max} = \gamma k_u$, where $\gamma$ is close to but less than 1.  Then:


\begin{equation}
 k_u /(1+x) = \gamma k_u
\end{equation}

Rewriting as  $x = 1/\gamma  - 1= C$ (where $C> 0$):

\begin{equation}
 \frac{ \phi_S  e^{-r'_u \xi}}{1- \phi_S - e^{-r'_u \xi}} =  1/\gamma  - 1 = C
\end{equation}

\begin{equation}
{ \phi_S  e^{-r'_u \xi}} = C ({1- \phi_S - e^{-r'_u \xi}} )
\end{equation}

Rearranging gives:

\begin{equation}
 \phi_S   =  \frac{C(1 - e^{-r'_u \xi})}{C  - e^{-r'_u \xi}} 
 \label{phi_S_RI}
 \end{equation}

This is an expression for the severity $ \phi_S $ at which $G_u = \gamma k_u$.

Rearranging, the expression for $\xi$ is: 


\begin{equation}
\xi   =  \frac{1}{r'_u}\log \frac{C  +\phi_S } {C(1 - \phi_S)}
 \label{RI_phi_S}
 \end{equation}

In the limit that  $G_u = k_u$, $C=0$.
 Equation \ref{phi_S_RI} yields $\phi_S = 0$,  and Equation \ref{RI_phi_S} yields $\xi \rightarrow \inf$.
 This is reassuring.


\subsection*{Estimating growth rates from timescales} 

Estimate the growth rates from the times for an isolated canopy to grow from $a k_u$ to $b k_u$, where $a$ and $b$ are constants less than 1.  
Substituting into the logistic equation solution (with $ G_uo = a k_u$):

\begin{equation}
    b k_u = \frac{k_u  a k_u }{ a k_u +(k_u- a k_u) e^{-r_u \tau}}
\end{equation}

where $\tau$ is the timescale to mature.
\begin{equation}
    b ({ a k_u +(k_u- a k_u) e^{-r_u \tau}) ={ a k_u }}
\end{equation}

\begin{equation}
    b ({ a  +(1- a ) e^{-r'_u t}) ={ a  }}
\end{equation}

\begin{equation}
r_u =   - 1/ \tau \log \big(\frac{a (1-b) }{(1- a )}\big)
\end{equation}


With $a$ = 0.1 and $b$ = 0.9, then $r_u \approx 4.5/ \tau$.

Estimate, growth rates from timescale to mature:

\begin{itemize}
\item conifer : $\tau$=30, $r=0.15$
\item shrubs : $\tau$=10, $r=0.45$
\item meadow : $\tau$=3, $r=1.50$
\item grassland : $\tau$=3, $r=1.50$
\end{itemize}


\subsection*{Non-dimensionalize}


 the equations for the upper and lower canopies:

\begin{equation*}
	 \frac{d G_u}{dt} =
	 r_u S^\beta G_u \bigg(1-\frac{G_u}{k_u}\bigg)
\end{equation*}


\begin{equation*}
	 \frac{d G_l}{dt} = r_l S^\beta G_l \bigg(1-\frac{G_l}{k_l}\bigg) - \alpha G_l G_u
\end{equation*}

\begin{equation*}
    \frac{d G_u}{dt} =
     \frac{r_u S^\beta}{r_u S^\beta} \frac{G_u}{dt} =
  	{r_u S^\beta} \frac{G_u}{d(r_u S^\beta t)} =
	r_u S^\beta \frac{G_u}{d \tau}
\end{equation*}

where $\tau = r_u S^\beta t$

\begin{equation*}
    \frac{d G_u}{d\tau} =
    G_u \bigg(1-\frac{G_u}{k_u}\bigg)
\end{equation*}


Let  $g_u = G_u/k_u$, or   $G_u =   k_u g_u$.  Then:

\begin{equation*}
  \frac{d   g_u}{d\tau} =
	 g_u \bigg(1 -  g_u \bigg)
\end{equation*}

Moving on to the lower canopy.  Substituting $\tau$ for $t$:

\begin{equation*}
	 r_u S^\beta \frac{d G_l}{d \tau } = r_l S^\beta G_l \bigg(1-\frac{G_l}{k_l}\bigg) - \alpha G_l G_u
\end{equation*}

Let  $g_l = G_l/k_l$, or   $G_l =   k_l g_l$.  Then:

\begin{equation*}
r_u   k_l  S^\beta \frac{d g_l}{d \tau } = r_l S^\beta k_l g_l \big(1-g_l \big) - \alpha g_l k_l g_u k_u
\end{equation*}

\begin{equation*}
r_u   S^\beta \frac{d g_l}{d \tau } = r_l S^\beta k_l g_l \big(1-g_l \big) - \alpha g_l k_l g_u k_u
\end{equation*}

\begin{equation*}
 \frac{d g_l}{d \tau } =\frac{ r_l}{r_u }  g_l \big(1-g_l \big) - \frac{\alpha k_u}{r_u  S^\beta}  g_l g_u
\end{equation*}

So our dimensionless groups are $r_l/r_u$ and $\alpha k_u / S_\beta r_u$, which are a generalized growth rate and competition, respectively.


We have several fixed parameters:  $k_u = 20$, $r_u = 0.25$, $S=0.21$.

$\alpha$ ranges from 0 (no competition) to 0.1,
$\beta$ ranges from 0 (no soil moisture feedback) to 1, $S^\beta$ ranges from 1  to 0.21

let  $r = r_l/r_u$ and $\phi =  \alpha k_u / S_\beta r_u$,

if $r_l$ ranges from 0.25 to 2.5,  $r$ ranges from 1 to 10.

\subsection*{Is Inertia a useful concept?}

What if we want to understand in terms of `intertia' relative to `competition'.


\begin{equation}
    \frac{1}{S^\beta}    \frac{dG_u}{dt} =  r_{u}  G_u \big(1-\frac{G_u}{k_u}\big)
\end{equation}

 \begin{equation}
    \frac{1}{S^\beta}    \frac{d G_l}{dt} = r_l  G_l \big(1-\frac{G_l}{k_l}\big) - \frac{\alpha}{S^\beta} G_u G_l,
\end{equation}

What is the best way to rewrite the expressions to convey intertia? 



\subsection{Dodo's model}

The growth of normalized tree biomass, $v$, in periods between fire occurrences is modeled with a logistic equation;
the growth rate is proportional to existing tree biomass and to the available resources $1-v$.

The temporal dynamics are expressed by stochastic equation:

\begin{equation}
\frac{dv}{dt} = \alpha v ( 1- v) - v F(t, v)
\label{dodo}
\end{equation}

$\alpha$ measures the inertia of the system, i.e. encroachment rate of trees and shrubs;
$F(t,v) $ is the disturbance caused by fires.

In mesic savannas, average fire frequency, $\lambda$, increases with increasing grass biomass.


Fire frequency is modeled as a linear function of $v$:
$\lambda  = \lambda_0 + bv$.

Like our model, 
the amount of tree vegetation killed by fires is proportional to the existing woody vegetation, $v$, and to the random variable $\omega$, measuring fire intensity.


\begin{equation}
\frac{dv}{dt} = \alpha v ( 1- v) - v F(t, v)
\end{equation}


Equation \ref{dodo} can be solved by transformation into an equation with no multiplicative noise, with $y = \log(v)$.

\begin{equation}
\frac{dy}{dt} = \alpha  ( 1- e^y ) -  F(t, v)
\label{dodo2}
\end{equation}

$F(t,v)$ is a. Poisson process of fire occurrence with rate $\lambda = \lambda_0 + b e^y$.

Equation \ref{dodo2} is integrated to get a steady state probability distribution of $y$.

%  \subsection*{Lower canopy equilibrium}
%
%As an alternative approach, what if the lower canopy is in dynamic equilibrium with the upper canopy?
%
%The lower canopy biomass is constant ($\frac{d G_l}{dt}  = 0$) if:
%
%\begin{equation}
% r_l S^\beta G_l \bigg(1-\frac{G_l}{k_l}\bigg) = \alpha G_{u,max} G_l
%\end{equation}
%
%
%Solving for $G_l$:
%
%\begin{equation}
%G_{l,eq}   = k_l\bigg(1 - \frac{ \alpha G_{u,max} }{ r_l S^\beta}\bigg)
%\label{G_l_eq}
%\end{equation}
%
%where $G_{l,eq} $ denotes the equilibrium value for $G_{l} $.
%
%From Equation \ref{G_uo}:
%
%\begin{equation}
%  G_{u,max} = \frac{ k_u}{(1-\phi_S)} \   \frac{1- \phi_S - e^{-r'_u \xi} }{1 - e^{-r'_u \xi}}
%\end{equation}
%
%Substituting, the equation is messy:
%
%\begin{equation}
%G_{l,eq}   = k_l\bigg(1 - \frac{ \alpha  k_u }{ r_l S^\beta}   \frac{(1- \phi_S - e^{-r'_u \xi}) }{(1-\phi_S)(1 - e^{-r'_u \xi)}}   \bigg)
%\end{equation}
%
%If $G_{l,eq}> 0 $, we expect the lower canopy biomass to approach this value in advance of each fire.
%However, if $G_{l,eq}= 0 $, lower canopy biomass may still be present because the system is out of equilibrium!
%
%Estimate a stability boundary as:
%
%
%\begin{equation}
%\frac{\alpha G_{u,max}}{r_l S^\beta} < 1
%\end{equation}
%
%Substituting for $G_{u,max}$:
%
%\begin{equation}
%\frac{\alpha }{r_l S^\beta} \frac{ k_u}{(1-\phi_S)} \   \frac{1- \phi_S - e^{-r'_u \xi} }{1 - e^{-r'_u \xi}} < 1
%\end{equation}
%
%And solving for severity as a function for $\xi$:
%
%
%\begin{equation}
%\phi_S < 1 - \frac{\alpha k_u e^{-r'_u \xi }}{\alpha k_u - r_l S^\beta (1- e^{-r'_u \xi })}
%\end{equation}



\newpage
\bibliographystyle{IEEEtranN}
\bibliographystyle{plain}
\bibliography{references}
\end{document}
